\documentclass[%
    hyperref={%
        colorlinks,
        linkcolor=sDarkBlue,
        urlcolor=sDarkBlue,
        citecolor=sDarkBlue
    },
    aspectratio=169
]{beamer}
\usetheme{sakura}
\ltjsetparameter{%
    jacharrange={%
        -2, % Exclude Greek and Cyrillic letters.
        -3  % Punctuations and Miscellaneous symbols.
    },
    alxspmode={`/,allow},
    alxspmode={`#,allow},
    alxspmode={92,allow}
}
\usepackage{luatexja-ruby}
\usepackage{breakurl}
\usepackage{booktabs}
\usepackage{ccicons}
\usepackage{subcaption}
\usepackage{fontawesome5}
\usepackage{amsmath,amssymb}
\usepackage{framed}
\usepackage{listings}
\usepackage{tikz-dependency}
\usepackage{wtref}
\renewcommand{\tablename}{表}
\renewcommand{\figurename}{図}
\newcounter{ngloss}
\renewcommand{\thengloss}{\alph{ngloss}}
\makeatletter
\newenvironment{gloss}[1][]{
    \refstepcounter{ngloss}
    \begin{table}[#1]
    \raggedright
}{
    \end{table}
}
\makeatother
\newref{tab}
\setrefstyle{tab}{prefix=表}
\newref{gloss}
\setrefstyle{gloss}{prefix=グロス(,suffix=)}
\newref{fig}
\setrefstyle{fig}{prefix=図}
\newref{math}
\renewcommand{\theequation}{\arabic{equation}}
\newcommand{\inlinecommand}[1]{{\ttfamily #1}}
\setrefstyle{math}{refcmd=(\ref{#1})}
\setbeamertemplate{caption}[numbered]
\setbeamertemplate{caption label separator}{:}
\setbeamertemplate{itemize items}{\raisebox{1.5pt}{\scalebox{0.4}{\ltjjachar`●}}}
\setbeamertemplate{bibliography item}{\insertbiblabel}
\usetikzlibrary{automata,positioning,arrows,calc}
\usepackage[%
  backend=biber,
  style=pecorarista,
  sorting=nyvt,
  urldate=long,
  natbib=true,
  maxnames=2
]{biblatex}
\addbibresource{demo.bib}
\DeclareMathOperator{\exponential}{exp}
\newcommand\header[1]{\multicolumn{1}{c}{\textbf{#1}}}
\newenvironment{quoteblock}{%
  \def\FrameCommand{%
    {\color{sLightGray}{\vrule width 3pt}}%
      \hspace{10pt}
  }%
  \MakeFramed {\advance\hsize-\width \FrameRestore}}%
{\endMakeFramed}
\lstset{%
  language={[LaTeX]TeX},
  basicstyle=\ttfamily\footnotesize,
  keywordstyle=\bfseries,
  texcsstyle=*\color{sRed}\bfseries,
  commentstyle=\color{sDarkBlue}
}
\renewcommand\refname{参考文献}
\renewcommand\appendixname{付録}
\title{Lua\TeX{}-jaとbeamerで研究発表用のスライドを作る}
\institute{所属}
\author{著者 太郎}
\date{{\number\year}年{\number\month}月{\number\day}日}
% https://tex.stackexchange.com/questions/2541/beamer-frame-numbering-in-appendix
\newcommand{\backupbegin}{
   \newcounter{framenumberappendix}
   \setcounter{framenumberappendix}{\value{framenumber}}
}
\newcommand{\backupend}{
   \addtocounter{framenumberappendix}{-\value{framenumber}}
   \addtocounter{framenumber}{\value{framenumberappendix}}
}
\begin{document}

    \maketitle

    \section{はじめに}
    \begin{frame}
        \frametitle{はじめに}
        このスライドは \faGithub\ \href{https://github.com/pecorarista/sakuratheme}{\ttfamily pecorarista/sakuratheme}のデモとして作ったものです.

        \bigskip

        そのため作り方を詳しく説明することはしませんが,
        コードはすべて上記のレポジトリに含まれているので気になる方は参照ください.
        \bigskip

        また言語学関連の話題の\LaTeX における扱い方を網羅的に知りたい方は
        \href{https://www1.essex.ac.uk/linguistics/external/clmt/latex4ling/}
        {LaTeX for Linguists}が参考になります.
    \end{frame}

    \section{基本}
    \begin{frame}
        \frametitle{表の挿入I}
        Beamerでは論文中の表のソースコードをほぼそのまま利用できて便利です.
        \begin{table}
            \centering
            \caption{表の例.}\tablabel{tab}
            \begin{tabular}{lrrr}
                \toprule
                \header{Model}   & \header{Precision} & \header{Recall} & \header{F1} \\
                \midrule
                \texttt{model-a} & 0.75 & 0.60 & 0.67 \\
                \texttt{model-b} & \textbf{0.80} & 0.70 & 0.75 \\
                \texttt{model-c} & 0.65 & \textbf{0.85} & 0.74 \\
                \texttt{model-d} & 0.78 & 0.78 & \textbf{0.78} \\
                \bottomrule
            \end{tabular}
        \end{table}
    \end{frame}

    \begin{frame}
        \frametitle{表の挿入II}
        特殊な記号を入力したい場合は\href{https://ctan.org/pkg/fontawesome5?lang=en}{fontawesome5}パッケージを利用すると便利です.
        \begin{table}
            \centering
            \caption{表の例.}\tablabel{tabfontawesome}
            \begin{tabular}{lcc}
                \toprule
                \header{Model}   & \header{Algorithm A} & \header{Algorithm B} \\
                \midrule
                \texttt{model-a} & \textcolor{sRed}{\faTimes} & \textcolor{sRed}{\faTimes} \\
                \texttt{model-b} & \textcolor{sRed}{\faTimes} & \textcolor{sOKGreen}{\faCheck} \\
                \texttt{model-c} & \textcolor{sOKGreen}{\faCheck} & \textcolor{sRed}{\faTimes} \\
                \texttt{model-d} & \textcolor{sOKGreen}{\faCheck} & \textcolor{sOKGreen}{\faCheck} \\
                \bottomrule
            \end{tabular}
        \end{table}
    \end{frame}


    \begin{frame}
        \frametitle{数式}
        \(X_n\)が\(X\)に確率収束する(converge in probability)とは,
        任意の\(\varepsilon > 0\)と\(\delta > 0\)に対して,
        ある自然数\(N\)が存在して,\(n \geq N\)を満たす任意の自然数\(n\)について
        \begin{align}
            P(\{\omega \in \varOmega \mid \lvert X_n(\omega) - X(\omega) \rvert > \delta \}) < \varepsilon \mathlabel{convprob}
        \end{align}
        が成り立つことをいう.

        \bigskip

        \mathref{convprob}を\(X_n \xrightarrow{P} X\)のように書くこともある.
    \end{frame}

    \begin{frame}[fragile]
        \frametitle{画像の引用}
        画像の挿入には\inlinecommand{\textbackslash includegraphics}コマンドを使います.
        Creative Commonsライセンスの作品には\href{https://ctan.org/pkg/ccicons}{\texttt{ccicons}}パッケージのアイコンを利用すると便利です.
        必要に応じて\inlinecommand{\textbackslash href\{{\rmfamily\mdseries\itshape uri}\}\{{\rmfamily\mdseries\itshape text}\}}で元のファイルへリンクを張ります.

        \bigskip

        \begin{figure}
            \includegraphics[width=0.4\textwidth]{cat.jpg}
            \caption{\href{https://www.flickr.com/photos/selda_eigler/8687127864/in/photolist-eeDNsC-qWFs4R-7CNDjJ-9c8DxY-eeDNhC-UCZ63T-dJNGUc-e5Nk39-988EVA-kUgwo-owDcVP-jQGjjt-5zkGTy-7WRCUo-b91XbZ-Mj8Ku-5pzwSA-9Bct2H-7CNHMY-7CJJMB-8MyEYn-9x45Mp-7JTq8M-ZrpGJ9-8fRht4-4SxVZT-5pzwjJ-ZsPJjL-aE44GL-dF6uWD-kqbHgM-5F373J-ZsQrVG-qyD7E9-ajyDPL-4WDvTp-KbDSc-5kCxD9-4MdeUo-pgDQcG-pPWrXD-662AFD-oTnC8k-apYceQ-nJSaaY-7CJLZv-7CJJMn-7CNFsU-XNMWkw-ccdtT9}{\emph{Cat} by Selda Eigler} \ccby.}
        \end{figure}

    \end{frame}

    \begin{frame}
        \frametitle{長め文章の引用}
        \href{https://ctan.org/pkg/framed}{\texttt{framed}}パッケージの\inlinecommand{leftbar}環境を使うと引用であることが分かりやすくなります.

        \begin{quoteblock}
    ἅπαν δὲ ὄνομά ἐστιν ἢ κύριον ἢ γλῶττα ἢ μεταφορὰ ἢ κόσμος ἢ πεποιημένον ἢ ἐπεκτεταμένον ἢ ὑφῃρημένον ἢ ἐξηλλαγμένον.

            \hfill \citetitle{poetics}
        \end{quoteblock}

        \begin{quoteblock}
            「あの森\ruby{琴}{ライラ}の宿でせう。
            あたしきつとあの森の中には、
            むかしの大きなオーケストラの人たちが集まつていらつしやると思ふわ。
            まはりには青い孔雀やなんかたくさんゐると思ふわ。」
            女の子が答へました。

            \hfill \citetitle{ginga}
        \end{quoteblock}
    \end{frame}


    \section{言語学}
    \subsection{グロス}
    \begin{frame}[fragile]
        \frametitle{グロス}
        大量に記載するのでなければ\href{https://ctan.org/pkg/gb4e?lang=en}{gb4e}ではなく\texttt{table}環境で十分だと思います。

        \bigskip

        \begin{gloss}
            \raggedright
            \begin{tabular}{lllll}
                (a) & {Это} & {учебник} & {русского} & {языка} \\
                % {Э'то} {уче'бник} {ру'сского} {языка'} \\
                    & {èto} & {učebnik} & {russk-ovo} & {jazyk-a} \\
                    & {this} & {textbook.\textsc{sg.nom}} & {Russian-\textsc{m.sg.gen}} & {language-\textsc{gen}} \\
                    & \multicolumn{4}{l}{``This is a textbook of the Russian language.''}
            \end{tabular}
            \glosslabel{ru}
        \end{gloss}

        \bigskip

        上の\glossref{ru}は\texttt{table}環境(をラップして定義した\texttt{gloss}環境)で作成しています.
        詳しくはこのスライドのソースコードを参照してください.
    \end{frame}

    \subsection{アラビア文字}
    \begin{frame}[fragile]
    \frametitle{アラビア文字I}
        もしアラビア文字を入力したければ\href{https://ctan.org/pkg/arabluatex?lang=en}{arabluatex}の利用をおすすめします.
    \begin{leftbar}
    \begin{lstlisting}[%
        language={[LaTeX]TeX},
        % asterisk for highlight backslash
        moretexcs={
            ltjsetparameter,
            arabicfont,
            newfontfamily,
            translitfont,
            SetTranslitFont,
            SetTranslitStyle,
            SetTranslitConvention
        },
        morekeywords={},
    ]
\usepackage{arabluatex}
\newfontfamily\arabicfont[%
  Script=Arabic, % enable ligatures
  RawFeature={%
    +anum, % use eastern arabic numerals
    +ss05} % put kasrah below shadda
]{Fira GO}
\newfontfamily\translitfont[Ligatures=TeX]{%
  TeX Gyre Termes
}
\SetTranslitFont{\translitfont}
\SetTranslitStyle{\itshape} % \upshape, \itshape
\SetTranslitConvention{arabica} % dmg, loc, arabica
    \end{lstlisting}
    \end{leftbar}
    \end{frame}

    \begin{frame}[fragile]
        \frametitle{アラビア文字II}
        ラテン文字で入力できるのでRTL(右から左への横書き)や合字に対応していないエディタでも編集できます.
        転写の方法は\inlinecommand{dmg}, \inlinecommand{arabica}, \inlinecommand{loc}の3種類から選べます.

        \begin{leftbar}
        \begin{lstlisting}[%
            language={[LaTeX]TeX},
            % asterisk for highlight backslash
            moretexcs={arb},
            morekeywords={arab}
        ]
\begin{arab}[fullvoc]
    'anta tatakallamu 'l-lu.gaTa
    'l-`arabiyyaTa jayyidaN!
\end{arab}
\arb[trans]{'anta tatakallamu
            'l-lu.gaTa 'l-`arabiyyaTa jayyidaN!}
        \end{lstlisting}
        \end{leftbar}

        \begin{arab}[fullvoc]
            'anta tatakallamu 'l-lu.gaTa 'l-`arabiyyaTa jayyidaN!
        \end{arab}

        \bigskip

        \arb[trans]{'anta tatakallamu 'l-lu.gaTa 'l-`arabiyyaTa jayyidaN!}
    \end{frame}

    \begin{frame}
        \frametitle{係り受け解析}
        係り受けの図を挿入するには\href{https://ctan.org/pkg/tikz-dependency}{\texttt{tikz-dependency}}を利用します.

        \bigskip

        \begin{figure}
            \begin{subfigure}[t]{0.38\textwidth}
                \centering
                \scalebox{.9}{%
                    \begin{dependency}[hide label]
                        \begin{deptext}[column sep=2\zw]
                            兵庫を \& 訪れた。 \\
                        \end{deptext}
                        \depedge{2}{1}{目的語}
                    \end{dependency}
                }
                \caption{文節単位・ラベルなし}\figlabel{dep-without-labels}
            \end{subfigure}
            \begin{subfigure}[t]{0.6\textwidth}
                \centering
                \scalebox{.9}{%
                    \begin{dependency}[text only label,label style={above}]
                        \begin{deptext}[column sep=2\zw]
                            兵庫  \& を  \& 訪れ \& た。 \\
                            {\scriptsize 固有名詞} \& {\scriptsize 接置詞} \& {\scriptsize 動詞} \& {\scriptsize 助動詞} \\
                        \end{deptext}
                        \depedge{1}{2}{格表示}
                        \depedge{3}{1}{目的語}
                        \depedge{3}{4}{助動詞}
                    \end{dependency}
                }
                \caption{単語単位・ラベルあり}\figlabel{dep-with-labels}
            \end{subfigure}
            \caption{文「兵庫を訪れた。」を係り受け解析し,図示したもの.}
        \end{figure}
        \figref{dep-without-labels}や\figref{dep-with-labels}のように参照することができます.
    \end{frame}

    \begin{frame}
        \frametitle{その他}
        箇条書きの項目が鉤括弧から始まるときの注意点
        \begin{itemize}
            \item こんにちは
            \item 「こんにちは」

                行頭の余白が大きい
            \item \leavevmode\inhibitglue 「こんにちは」

                \texttt{\textbackslash item \textbackslash leavevmode\textbackslash inhibitglue} で余白を調整
        \end{itemize}

        \bigskip

        参照:\href{http://doratex.hatenablog.jp/entry/20140714/1405302796}{「TeX Live 2014のpTeX系列における\textbackslash inhibitglueの仕様変更」}
    \end{frame}

    \backupbegin

    \begin{frame}[%
        % allowframebreaks,  % Enable `allowframebreaks` if the number of pages is larger than 1.
        plain
    ]{参考文献}

    \printbibliography[%
        title={参考文献}  % for PDF bookmark
    ]

    \end{frame}

    \backupend

\end{document}
