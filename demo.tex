\documentclass[11pt]{beamer}
\usepackage[sort]{natbib}
\usepackage{framed}
\makeatother
\usepackage[no-math]{fontspec}
\usepackage{luatexja-ruby}
\ltjsetparameter{%
    jacharrange={%
        -2, % Exclude Greek and Cyrillic letters.
        -3  % Punctuations and Miscellaneous symbols.
    },
    alxspmode={`#,allow}
}
\usetheme{sakura}
\title{主題}
\subtitle{副題}
\institute{所属}
\author{名前}
\date{{\number\year}年{\number\month}月{\number\day}日}
\hypersetup{%
    unicode=true,
    backref=true,
    hidelinks=true
}
\usepackage{gb4e,cgloss4e}
\noautomath%
\let\eachwordtwo=\sffamily
\let\eachwordthree=\sffamily
\begin{document}

\begin{frame}
    \nocite{demo}
    \maketitle
\end{frame}

\section{言語学関連}
\subsection{例文と対訳}
\begin{frame}
\frametitle{番号付きの例文}
    \begin{exe}
        \ex%
        \glll%
        {憎悪} {の} {炎} {が} {燃え上がった} \\
        {zouo} {no} {honoo} {ga} {moeagatta} \\
        {hatred} {\textsc{gen}} {flame} {\textsc{nom}} {burn\_up-\textsc{pst}} \\
        \trans%
        ``The flame of hatred burned up.''
    \end{exe}
\end{frame}

\begin{frame}
    \frametitle{長め文章の引用}
    \begin{leftbar}
        「あの森\ruby{琴}{ライラ}の宿でせう。
        あたしきつとあの森の中には、
        むかしの大きなオーケストラの人たちが集まつていらつしやると思ふわ。
        まはりには青い孔雀やなんかたくさんゐると思ふわ。」
        女の子が答へました。

        \hfill 宮澤賢治「銀河鉄道の夜」
    \end{leftbar}
\end{frame}

\begin{frame}
\frametitle{日本語で書かれた参考文献の引用}
\texttt{{\textbackslash}cite\{key\}}で\cite{demo}や\cite{japanese}のように引用されます.
\end{frame}

\begin{frame}[allowframebreaks]
\frametitle{参考文献}
\begingroup
\scriptsize
    \setbeamertemplate{bibliography item}[triangle]
    \bibliographystyle{jecon}
    \bibliography{demo}
\endgroup
\end{frame}

\end{document}
